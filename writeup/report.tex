\documentclass[12pt]{article}
\usepackage{csvsimple}
\usepackage{amsmath}
\usepackage{a4wide}
\usepackage{graphicx}
\usepackage{natbib}
\usepackage{longtable}


\usepackage[scaled]{helvet}
\usepackage[T1]{fontenc}

\usepackage{setspace}
\onehalfspacing

\usepackage[style=base,font=onehalfspacing]{caption}

\usepackage[left=2cm,right=2cm,top=2cm,bottom=2cm]{geometry}

\title{Prioritising species sampling for genetic biobanking}
\author{Eamonn Murphy}

\renewcommand\familydefault{\sfdefault}
\begin{document}
	\begin{titlepage}
		\newcommand{\HRule}{\rule{\linewidth}{0.6mm}}
		\begin{center}
			\vspace{\baselineskip}
			\huge
			Prioritising Species for Genetic Biobanking
			
			\HRule \\
			\large
			\vspace{5ex}
			Authored by Eamonn Murphy
			
			\medspace
			
			Supervised by Dr. James Rosindell and Prof. Francisco Pelegri
			
			\vspace*{\fill}
			
			Submitted August 2022
			
			\vspace*{\fill}
			\normalsize
			A thesis submitted in partial fulfilment of the requirements for the degree of
			Master of Science at Imperial College London
			\vspace{2ex} \\
			Submitted for the MSc in Computational Methods in Ecology and Evolution
	\end{center}

	\end{titlepage}
	
	\pagebreak
	
	\section{Declaration}
	I declare the work presented in this thesis is my own and that I am the
	sole author. The data was obtained from the published literature. Data that
	had been cleaned to match species names between IUCN categories and
	phylogenetic trees
	was provided to me by Dr. Rikki Gumbs. I obtained and cleaned data on estimated
	conservation spending. Any further data processing or cleaning was performed
	by me. I developed the mathematical models used with the
	guidance of my primary supervisor, Dr. James Rosindell. I developed the analyses
	and methodology used under the supervision of Dr. James Rosindell, and with
	the input of my co-supervisor, Prof. Francisco Pelegri.
	
	\pagebreak
	
	\section{Abstract}
	Global biodiversity is under threat from mass extinctions
	and population declines, accelerated by changing climates and
	environments. Traditional conservation methods will likely be
	insufficient to prevent the loss of irrecoverable evolutionary
	and genetic diversity. Interspecies cloning from frozen
	biological samples offers the promise of revitalising the genetic
	diversity of threatened populations, and even of
	restoring species from extinction. Preserving the necessary samples
	now will be vital to utilising these methods,
	and we should focus our efforts on those samples which have
	the highest chance of viability. A Priority Score was developed
	which calculates the probability that a sample will be useful
	for restoring a species from extinction, through interspecies
	cloning. It is based
	on the probability
	that the species will go extinct, alongside the survival
	of a compatible donor species. Through exploring a range of
	possibilities for the
	evolutionary distance at which a species can be compatible,
	I discovered that this has a large impact on which species are
	prioritised.
	The use of an optimised approach to species for biobanking
	markedly improved the number of potential restorations
	enabled, and similiar methods to prioritise conservation spending
	were shown to reduce extinctions at a given level of spending. It should
	be a priority for conservationists globally to ensure samples of our
	biodiversity are effectively preserved for the long-term, before
	it is irretrievably lost, and my results show that these efforts
	can be effectively targeted to maximize outcomes.
	
	\pagebreak
	
	\section{Introduction}

	Scientists believe that we may be living through the time of the Sixth 
	Great Mass Extinction \citep{barnoskyHasEarthSixth2011}. Anthropogenic changes to the natural environment, such 
	as climate change and pollution, have already driven many species extinct and 
	threaten to cause the extinction of many more in the coming centuries \citep{ceballosVertebratesBrinkIndicators2020}. 
	Traditional conservation strategies focus on mitigating the impacts of human
	activities, as well protecting some of our most vulnerable and 
	threatened species. However, this may not be sufficient to protect some of the most
	vulnerable species, as these approaches are often costly to implement, and 
	the political will can be lacking. Another approach, previously confined to
	science fiction stories, but made possible by advances in 
	biotechnology, is to clone individuals from threatened (and potentially even
	extinct) species using stored biological material \citep{loiGeneticRescueEndangered2001,leeCanCloningEndangered2001}.
	In order to make this approach possible, we must biobank samples from threatened species now, before it is too late.
	
	Traditional conservation methods include \textit{in situ} and 
	\textit{ex situ} activities \citep{angelesChallengesDevelopmentBiodiversity2022}.
	\textit{In situ} activities include habitat
	protection and restoration, and protecting species from poaching and other
	harmful human activities \citep{mccarthyFinancialCostsMeeting2012}.
	\textit{Ex situ} activities such as breeding
	programmes at zoos and aquariums have also played an important role.
	The goal of breeding
	programmes has been to increase the size of the population, while
	maintaining its genetic diversity
	\citep{farquharsonOffspringSurvivalChanges2021}. However, this can be 
	difficult, particularly when the source wild population is small, or if the species has
	difficulties with breeding in captivity (meaning most successful
	offspring derive from a few individuals). Furthermore, breeding in captivity can
	result in reduced fitness in the wild through selection for traits like tameness
	\citep{arakiCarryoverEffectCaptive2009,fordSelectionCaptivitySupportive2002}.
	
	Maintaining genetic diversity within a population is key to its long-term health and
	survival. Existing genetic diversity is important to allow for selection under
	evolutionary pressures \citep{harmonConservationSmallPopulations2010}. This allows a population to adapt to threats, such as pathogens or changing environments
	\citep{mccallumTasmanianDevilFacial2008}. Evolutionary rescue, where a population
	avoids extinction through selection
	\citep{bellEvolutionaryRescue2017}, relies on standing genetic diversity that
	exists within the population. Evolutionary rescue must occur too quickly
	for mutation to play a significant role, so the selected traits must already
	be present.
	Greater diversity of traits or alleles
	increases the chance that a positive adaptation can occur.
	
	The effective population size ($N_e$) is a measure of
	population size which also takes into account the relatedness between
	individuals, thus serving as a good measure of the total genetic diversity
	\citep{frankhamEffectivePopulationSize1995}.
	A threshold can be created
	using the effective population size, below which populations enter
	what is known as the extinction vortex \citep{harmonConservationSmallPopulations2010}. This threshold is usually set at an $N_e$ of
	around 50. Upon entering the extinction vortex, populations not only struggle to
	adapt to environmental changes, but begin to suffer from
	inbreeding, which can lead to decreased fitness and reduced fertility
	\citep{nabutanyiModelsEcoEvolutionaryExtinction2021}. It is
	difficult to rescue a population once it has entered the extinction vortex, as genetic
	drift will tend to overpower mutation at such small population sizes and genetic
	variation will continually be lost \citep{harmonConservationSmallPopulations2010}.
	One current method for improving genetic diversity is genetic rescue,
	where new alleles are introduced to a population through managed immigration
	of individuals from an outgroup
	\citep{whiteleyGeneticRescueRescue2015}. However, this method can only be used to
	improve the prospects of sub-populations of a species, and not the species as a
	whole, as there is no traditional method for improving genetic diversity across
	an entire species.
	
	Genetic diversity could potentially be restored to a vulnerable
	population through cloning of frozen cell samples. Creating clones of existing
	individuals, brought into the public consciousness through Dolly the sheep
	\citep{campbellSheepClonedNuclear1996}, has been possible now for a few decades.
	It involves fusing nuclei from cultured cell lines, which initially needed to be
	embryonic, with enucleated
	oocytes (egg cells which have had their nuclei removed, often
	through a combination of mechanical and chemical means
	\citep{vajtaHighlyEfficientReliable2006}).
	The fused oocyte begins to 
	multiply, forming a blastocyst, upon which it can be implanted into a host
	womb and carried through pregnancy. Further developments allowed for
	use of somatic cell lines as nuclear donors, where the somatic
	nucleus is reprogrammed by the cytoplasmic factors present in the egg
	to become a zygotic nucleus \citep{shapiroPathwaysDeextinctionHow2017}.
	Individuals of endangered
	species were soon successfully cloned from cell lines derived from live
	adults \citep{loiGeneticRescueEndangered2001}. The cloning of the mouflon
	(\textit{Ovis orientalis musimon}) by \citet{loiGeneticRescueEndangered2001}
	was achieved using an interspecies
	approach (interspecies Somatic Cell Nuclear Transfer, iSCNT
	\citep{wiselyRoadMap21st2015}). The nuclei obtained from mouflon cell lines were
	fused
	with oocytes obtained from domestic sheep 
	(\textit{Ovis aries}), and the developing blastocyst was implanted into a sheep
	to carry to term. Importantly, the cells which the donor nuclei were
	derived from were non-viable (could not be induced to divide),
	and yet successfully
	produced clones.
	
	Cloning of individuals from cells derived from live animals can be useful, but
	cloning from frozen samples allows reintroduction of lost genetic diversity.
	This approach has recently been used
	to clone an individual of the black-footed ferret (\textit{Mustela nigripes}),
	from cells that had been frozen for over 30 years
	\citep{frittsConservationFirstCloned2022}.
	The black-footed ferret is an endangered species native to the 
	USA, and a breeding programme has attempted
	to increase the numbers of the species, in order
	to complement the wild population \citep{sandlerEthicalAnalysisCloning2021}.
	However, despite the best efforts of the breeding
	programme, genetic diversity within the population has been continually declining.
	Frozen cell samples, taken in 1987, contained genetic variation
	absent from the current population \citep{frittsConservationFirstCloned2022}.
	For this reason, it was decided to attempt an iSNCT, using the domestic ferret
	(\textit{Mustela putorius}) as a source
	of oocytes and to host embryos \citep{wiselyRoadMap21st2015}.
	A cloned individual was successfully birthed, and
	remained alive after 90 days \citep{sandlerEthicalAnalysisCloning2021}.
	
	As well as reintroducing genetic diversity to threatened populations,
	interspecies cloning from frozen samples may be used to revive species
	from extinction. This was achieved by \citet{folchFirstBirthAnimal2009},
	who successfully cloned the recently extincted bucardo
	(\textit{Capra pyrenaica pyrenaica}) from frozen cell samples, using the
	Spanish ibex (\textit{Capra pyrenaica}) as the source of enucleated oocytes
	and foster mothers. However, the single successfully cloned individual
	died at two days old. This gives an indication of the difficulty of
	this process, as it is the only known successful birth of a cloned
	individual from an extinct species to date
	\citep{shapiroPathwaysDeextinctionHow2017}.
	
	As the examples of the black-footed ferret and bucardo show,
	the interspecies restoration approach
	relies upon biobanked samples preserving the genetic diversity of the species.
	This necessitates storing samples we might wish to use in the future now, particularly given the current
	rates of extinction and population decline. Under a scenario of limited resources,
	a key consideration is which species to biobank first \citep{harwoodDevelopingImplementingPrioritisation2021}, or prioritise for
	other aspects of developing restoration programmes such as cell
	line development and breeding programmes.
	Often, conservation resources are focused on species with high attractiveness or
	cultural appeal \citep{gunnthorsdottirPhysicalAttractivenessAnimal2001}. Although this approach can help to drive resources
	towards conservation, it does not optimise the use of the resources we have, 
	and can result in oversights that leave behind species of high worth on other
	metrics, such as ecological importance or phylogenetic diversity. This has been
	the motivation of approaches such as EDGE or HEDGE scores, which prioritise conservation 
	resources based on aims such as maximising retained phylogenetic diversity \citep{isaacMammalsEDGEConservation2007,steelHedgingOurBets2007}.
	
	Mirroring these approaches, we can attempt to prioritise species for biobanking
	by utilising data on phylogenetics and extinction risk.
	The focus in this work was on biobanking material for use in interspecies
	restoration methods. In order for an
	interspecies approach to be possible, we need biobanked material from the target
	species, as well as the presence of some sufficiently closely related species to act
	as a donor of egg cells and a host for embryos. Therefore, the highest priority species
	for biobanking are those which have a high probability of extinction, alongside a high
	probability of the survival of a potentially compatible donor species. The current best test of
	relatedness is phylogenetic distance, particularly in a general, cross-taxa approach. Therefore, prioritisation was performed
	by combining IUCN Red List data with a phylogenetic tree of the taxon of 
	interest.
	
	This project aimed to create a metric for prioritising species for
	biobanking, based on the potential usefulness of the material for interspecies
	restoration methods. Factors like the distance at which two species can still
	be compatible for interspecies cloning could affect the order of
	prioritisation. I aimed to assess some of these factors, as well as running
	simulations to benchmark the effectiveness of the developed prioritisation
	methods. I also wished to compare
	traditional conservation spending against biobanking, by simulating the
	effect on extinctions of various levels of conservation spending, based on
	estimates from expert surveys
	\citep{mccarthyFinancialCostsMeeting2012,nunesPriceConservingAvian2015}.

	%still need to discuss - 
	% 	where to set the thresholds / how are these set
	%	the use of conversions from IUCN status to extinction probabilities
	
	\pagebreak
	
	\section{Methods}
	%\begin{table}
	%	\begin{tabular}{|c|c|c|c|}
	%		\hline
	%		Cryovial Volume (mL) & Storage Temp ($^\circ C$) & Cost per Sample Year 1 & Annual Cost Ongoing \\
	%		\hline
	%		2.0 & -20 & £1.08 & £0.05 \\
	%		 & -80 & £1.18 & £0.15 \\
	%		 & -196 & £1.25 & £0.22 \\
	%		\hline
	%		0.5 & -20 & £0.27 & £0.01 \\
	%		 & -80 & £0.29 & £0.04 \\
	%		 & -196 & £0.31 & £0.06 \\
	%		\hline
	%	\end{tabular}
	%	\caption{Estimated material costs for sample storage in the NHM Molecular Collections Facility}
	%\end{table}

	\subsection{Priority Score Calculation}\label{ps_method}
	In order to generate a descending order list for prioritising species to biobank,
	a Priority Score was calculated for each species. This Priority Score is the probability
	that the biobanked material may be useful for interspecies restoration approaches, i.e.
	that the species is extinct, and that a compatible relative species survives.
	We can define a compatible species as one that shares a most recent common
	ancestor (MRCA) with the species of interest within $T$ million years ago.
	Therefore, there is a set of sufficiently close relatives, $A_i(T)$,
	for each species, varying depending on the value of $T$ (see
	Equation \ref{set}). We can then define the Priority Score as $P_i(T)$, equal
	to the probability of extinction of the species of interest ($P_{ext}(i)$) multiplied by
	the probability of a compatible species surviving (see Equation \ref{equation}). 
	This can be calculated as 1 minus
	the probability of all compatible species going extinct,
	which equals the product of the probabilities
	of extinction of each compatible species $a$ from the previously defined set $A_i(T)$.
	This definition also holds for the case where there are no compatible species, as by definition, the product of an empty set is 1, producing a Priority Score of 0.
	
	\begin{align}
		P_i(T) &= P_{ext}(i) * [1 - \prod_{a \in A_i(T)} P_{ext}(a)] \label{equation}\\
		A_i(T) &= \{j \in S | MRCA(j, i) \le T\} \label{set}
	\end{align}

	Priority Scores were calculated for two taxa of broad interest,
	mammals and birds. Mammals were of interest as attempts have been made to use
	interspecies methods to assist conservation within this class, such as the
	black-footed ferret. Birds were used as a comparison taxon, and also because there are data
	available on estimated costs of conservation for bird species. In order to produce the set $A_i(T)$
	for each species, time since divergence was calculated using phylogenetic
	trees. 100 different phylogenetic trees were used for each taxon, to avoid biasing the results to any
	particular phylogeny 
	\citep{uphamInferringMammalTree2019,jetzGlobalDiversityBirds2012}. %citep source of trees from Rikki
	The Priority Scores for all species were calculated for each phylogeny,
	and the mean across all 100 trees was used as the final
	Priority Score. A number of different transformations for the probabilities of species
	extinction, based on the IUCN Risk Categories, are available, as shown in Table \ref{ext_prob}.
	Each of these transformations was used to
	calculate a different set of Priority Scores, changing the probabilities of extinction as
	appropriate for each model. To explore the effect of varying $T$ on the distribution of Priority
	Scores, I calculated Priority Scores for a range of values of $T$, from 1
	to 20 million years since the MRCA. To visualise these distributions effectively,
	the number of high priority species for each calculation were counted. High priority
	species were defined as those with a Priority Score of > 0.95 (implying a 95\%
	chance of being useful for interspecies restoration under these assumptions).
	This threshold was chosen as it is close to the extinction probabilities
	of the Critically Endangered species for the 50 year and pessimistic scenarios,
	and the extinction probabilities of the Endangered species for the 500 year
	scenario (see Table \ref{ext_prob}).
	
	\subsection{Extinction Simulations}\label{sim_methods}
	Simulated extinctions were carried out for two reasons, to deduce the usefulness of the prioritisation
	method compared to random sampling, and to generate cost estimates of saving species from
	extinction using traditional conservation techniques. For all simulations, the IUCN 50
	year transformation of extinction probabilities \citep{mooersConvertingEndangeredSpecies2008} was used, alongside
	a $T$ of 10 million years. Simulations were performed by randomly drawing a number from the 
	uniform distribution between 0 and 1 for each species,
	and then comparing this to their extinction probability. If the randomly drawn number 
	was less than the extinction probability, that species was deemed to have gone extinct, and other species were determined to have survived.
	
	To benchmark the usefulness of the prioritisation method for biobanking, a restoration simulation
	was performed. A list of biobanked species of a given length was generated,
	either selected by one of three methods: random sampling,
	descending order based on Priority Score, and descending order
	based on extinction risk category, i.e. all Critically
	Endangered species biobanked first, then all Endangered species, etc.. Then, for each biobanked
	species, I checked if they had reached extinction in the simulation; if they had, then I 
	checked if any compatible species from the set $A_i(T)$ had survived.
	If both of these conditions were met, the species was considered capable of "restoration",
	i.e. using interspecies cloning for de-extinction of the species.
	
	Simulations were also used to estimate the number of extinctions prevented by a given level
	of conservation spending. In this case, before running the simulations, a
	number of species were downlisted (moved down one IUCN extinction
	risk category), Then, simulations
	were ran as described with the new extinction probabilities, and the number of extinctions observed
	was recorded. To determine which species would be downlisted, a certain budget was set for each
	simulation. Species were selected either by random sampling, or by an optimised method
	that selected the species representing the best value for money,
	in descending order. Value for money was determined by dividing the cost of conservation
	by the difference in probability of extinction between the two categories, e.g. for
	a Critically Endangered species, this would be 0.55,
	(0.97, the extinction probability for Critically
	Endangered species, minus 0.42, the extinction probability for Endangered
	species).
	
	\subsection{Data Sources}
	Extinction risk categories for mammal and bird species were
	sourced from the IUCN Red List 2020 \citep{iucnIUCNRedList2021, iucnIUCNRedList2012}. This was combined with 100 phylogenetic trees for each class, taken from \citet{uphamInferringMammalTree2019}
	for mammals, and \citet{jetzGlobalDiversityBirds2012} for birds, respectively.
	Each species was assigned a probability of extinction based on its
	extinction risk category, according to the transformations developed by \citet{mooersConvertingEndangeredSpecies2008} (see Table \ref{ext_prob}).
	
	\begin{table}[b!]
		\centering
		\caption{Transformations of IUCN extinction risk categories to probabilities
			of extinction, from \citet{mooersConvertingEndangeredSpecies2008}.
			Three transformations are presented, two projected from the
			criteria for the IUCN Red List \citep{iucnIUCNRedList2012},
			and one arbitrary pessimistic model of species extinction.
		}\label{ext_prob}
		\begin{tabular}{|c|c c c c|}
			\hline
			Status & Abbreviation & IUCN 50 years & IUCN 500 years & Pessimistic \\
			\hline
			Least Concern & LC & 0.00005 & 0.0004 & 0.2 \\
			Near Threatened & NT & 0.004 & 0.02 & 0.4 \\
			Not Available & NA & 0.004 & 0.02 & 0.4 \\
			Data Deficient & DD & 0.004 & 0.02 & 0.4 \\
			Not Evaluated & NE & 0.004 & 0.02 & 0.4 \\
			Vulnerable & VU & 0.05 & 0.39 & 0.8 \\
			Endangered & EN & 0.42 & 0.996 & 0.9 \\
			Critically Endangered & CR & 0.97 & 1 & 0.99 \\
			Extinct in the Wild & EW & 1 & 1 & 1 \\
			\hline
		\end{tabular}
	\end{table}
	
	Data was also obtained on the estimated conservation spending needed for 206 different bird species, from \citet{mccarthyFinancialCostsMeeting2012}. This data was collected via expert surveys, and was their best estimate of the cost to downlist that bird species,
	which means to move it from one IUCN risk category to the next category
	"down", e.g from Critically Endangered
	to Endangered. It was formatted as the estimated spending per year, in US \$, needed over 10 years to result in species downlisting. To account for inflation, amounts were converted from January 2012
	US \$ to May 2022 US \$ using the US Bureau of Labour Statistics CPI Inflation Calculator
	\citep{CPIInflationCalculator}. % need to citep the website from Zotero, https://www.bls.gov/data/inflation_calculator.htm
	Mismatches between the species names in this data and the 2020 IUCN data were resolved using Global Names Verifier \citep{mozzherinGnamesGnverifierV12022}. Any species without a clear synonymic or misspelling match were removed from the data. Species that had changed IUCN extinction risk category
	during the intervening period were also removed from the data. This left a total of 166 species to use for the analysis, and the final cleaned data is available at the location cited in the Data and Code Availability Statement, Section
	\ref{availability_statement}. 
	To extend these estimates to the other 1325 species included in the IUCN Red List dataset I was 
	using, each species without a cost was randomly assigned a cost from one of the estimates within
	the same extinction risk category, i.e. a Critically Endangered species
	without an estimate of the cost of conservation was randomly assigned
	one of the estimates for the Critically Endangered species.
	
	
	\subsection{Code}
	Code for this project was written and run using R version 4.2.1 \citep{rcite}
	on a Unix-based operating system.
	High-throughput scripts were
	run using the Imperial College Research Computing Service.
	Libraries used
	include ape Version 5.6-2 \citep{paradisApeEnvironmentModern2019},
	dplyr Version 1.0.9 \citep{dplyr},
	stringr Version 1.4.0 \citep{stringr}, and stringdist \citep{stringdist}.
	
	\pagebreak
	
	\section{Results}
	\subsection{Prioritising Species for Biobanking}
	Figures \ref{mammal_hist} and  \ref{bird_hist} display distributions of
	Priority Scores and probability of potentially compatible relatives
	surviving, for mammals and birds respectively. Each of the
	three extinction probability transformations is plotted separately
	(IUCN 50 year, IUCN 500 year and pessimistic),
	and all calculations for these
	figures were performed using
	a $T$ of 10 million years to the MRCA. Priority Scores are in the form
	of a probability, ranging between 0 and 1. There is a peaked
	distribution for all models and taxa, with most Priority Scores falling into
	clearly separated buckets. This distribution is similar to the underlying
	shape of the distribution of probabilities of extinction (see Table \ref{ext_prob}
	for these values), which has a discrete value for each extinction risk
	category. We can see that the distribution of Priority Scores for the
	pessimistic model differs significantly from the others, both in isolation
	and in comparison to the underlying extinction risk values for this model.
	The probability of a compatible relative surviving follow a very asymmetric
	bimodal distribution, with a very large peak near a probability of 1 and
	a smaller peak near a probability of 0. The pessimistic
	model has a more even distribution, with a lower proportion of the probabilities
	falling very close to 1. Distribution of the probabilities of a compatible
	relative surviving
	is more even across models for birds (Figure \ref{bird_hist}) compared to
	mammals (Figure \ref{mammal_hist}), with a greater proportion of probabilities
	lying away from the extremes.
	
	
	\begin{figure}
		\centering
		\includegraphics{../results/mammal_combined_hist.pdf}
		\caption{Distribution of Priority Scores (A-C) and probabilities of
			compatible relatives surviving (D-F) for \textbf{mammals}, at a $T$
			of 10 million years since the MRCA. Each column was calculated using a
			separate model; the IUCN 50 year model for A and D, the IUCN 500 year
			model for B and E and the pessimistic model for C and F.
			Histograms have 20 bins and are scaled to the same axes.
			}\label{mammal_hist}
	\end{figure}
	
	\begin{figure}
		\centering
		\includegraphics{../results/bird_combined_hist.pdf}
		\caption{Distribution of Priority Scores (A-C) and probabilities of
			compatible relatives surviving (D-F) for \textbf{birds}, at a $T$
			of 10 million years since the MRCA. Each column was calculated using a
			separate model; the IUCN 50 year model for A and D, the IUCN 500 year
			model for B and E and the pessimistic model for C and F.
			Histograms have 20 bins and are scaled to the same axes.
			}\label{bird_hist}
	\end{figure}

	%\begin{figure}
	%	\includegraphics[scale=0.1]{../results/both_colored_trees.png}
	%	\caption{}
	%\end{figure}

	\subsection{Effect of $T$ Upon Priority Scores}
	The change in the number of high priority species (Priority Score of > 0.95)
	with variation in the value of $T$, for each of the three different models, is
	shown in Figure \ref{thresh_viz}. For all models and taxa, the number of high
	priority species increases as we increase the value of $T$. This increase
	is sharpest at lower values of $T$, and as the number of high priority
	species approaches the number of Critically Endangered species (IUCN 
	50 year model and pessimistic model) or Endangered species (IUCN 500 year
	model), this increase tapers off, which reflects the total number of species
	whose extinction probability exceeds the high priority threshold for
	each model. The relative or percentage increase in the number of high
	priority species as $T$ increases
	is greater for the IUCN 500 year and pessimistic models.
	
	\begin{figure}
		\centering
		\includegraphics{../results/threshold_viz.pdf}
		\caption{The number of species with a high Priority Score (> 0.95)
			against the threshold used for potentially compatible relatives.
			A Priority Score of > 0.95 implies a probability of >
			95\% that the material may be useful for interspecies restoration
			in the future. IUCN 50 year, IUCN 500 year and pessimistic models
			are plotted separately for mammals (A) and birds (B). Dashed line
			indicates the number of critically endangered species within each
			taxon. The upper bound of the y-axis is the combined total of
			endangered and critically endangered species within each taxon.
			}\label{thresh_viz}
	\end{figure}
	
	\subsection{Simulating the Usefulness of Biobanked Material}
	We compared the number of species that could potentially be restored using
	a random and two optimised biobanking approaches for interspecies restoration,
	visualised in Figure \ref{rest_sims}. The optimised biobanking approaches
	result in a far greater number of potential species restorations
	compared to random biobanking, for both birds and mammals. The category-based
	optimisation appears to result in a slightly superior number of restorations
	when compared to the Priority Score based optimisation. The number
	of restorations for the optimised biobanking approaches have a clearly stepped
	shape for both mammals and birds. The transitions between these steps
	correspond to the numbers of prioritised species for each extinction risk
	category in turn, i.e. we can see the transition point from where Critically
	Endangered species are being biobanked to Endangered species, and so on.
	The distributions have quite similar shapes for the two taxa, especially
	when taking into account the underlying numbers of species in each category.
	The random biobanking approach is, however, noticeably more successful in mammals
	than in birds.
	
	\begin{figure}
		\centering
		\includegraphics{../results/restoration.pdf}
		\caption{The number of potential species restorations, comparing
		a random and two optimised biobanking approach for both mammals and birds.
		The optimised approach (red dots) used a descending order list based on
		Priority Scores, whereas the category based optimisation chose the species
		with the highest available extinction risk category, i.e. all Critically
		Endangered species were biobanked first, followed by Endangered
		and so on.
		These simulations were performed using the IUCN 50 year model of
		extinction probabilites, and with a $T$ of 10mya.}\label{rest_sims}
	\end{figure}

	
	\subsection{Conservation Costs and Extinction}
	The likely number of extinctions prevented for a certain level of
	conservation spending on birds was simulated, according to the
	methods set out in Methods Subsection \ref{sim_methods}. The optimised
	approach to conservation spending yielded clearly superior results,
	visualised by Figure \ref{expense_sims}. The spending required to
	change the number of extinctions from baseline was approximately
	66 times greater for the random approach, almost two orders of
	magnitude (\$6 billion vs. \$90 million). We can see from Figure
	\ref{expense_sims}B that the number of extinctions using the random approach
	has a linear relationship to spending, up to a certain amount, whereas
	the optimised approach has a curved relationship.
	
	
	\begin{figure}
		\centering
		\includegraphics{../results/log_expense_sims.pdf}
		\caption{The number of simulated extinctions among birds, against
			the amount of conservation spending in US \$ (B), and with
			spending on a log scale (A). Each dot
			represents a simulation of extinctions using the IUCN 50 year
			model of extinction probabilities, given a number of
			downlistings determined by the amount of conservation
			spending. Species choice for downlisting is determined
			either randomly, or by greatest reduction in extinction
			probability per dollar spent (optimised).
			}\label{expense_sims}
	\end{figure}

	\pagebreak
	
	\section{Discussion}
	To the best of my knowledge, there are no previous publications in the
	primary literature to create metrics for
	biobanking prioritisation. However, there have been theses written which 
	concentrated
	on this task, and these attempts focused mainly on aspects such as
	samples currently existing, and preserving phylogenetic diversity 
	\citep{mooneyValueExSitu2021,harwoodDevelopingImplementingPrioritisation2021}.
	% check out these citations
	This research is potentially
	the first to analyse sample prioritisation by
	a direct measure of the sample's potential usefulness in the future,
	namely its use for interspecies restoration. The aim of this research
	was to create a metric to prioritise species for biobanking for this
	use, and to analyse the factors which affected the order of priority,
	as well as test the applicability of this prioritisation to the
	real world.
	
	The distribution of Priority Scores looks similar to the underlying
	distribution of extinction risks for a $T$ of 10mya, with some
	key differences. The Priority Score for a given species must be less
	than its underlying probability of extinction, because
	in the case that the species goes extinct, the possibility of restoration
	still relies upon another factor, a potentially compatible relative
	surviving. For most species with close relatives, their Priority Score under
	the 50 year and 500 year models is only
	mildly affected by the probability of a close relative surviving, as these
	probabilities are very clustered near 1 (Figure 
	\ref{mammal_hist} and \ref{bird_hist}, D and E).
	Most of the apparent difference in the distribution
	comes from species who have no close relative with a MRCA of $<$ 10mya,
	which is the smaller peak of probabilities near 0 .
	However, the distribution for the pessimistic model is more evenly spread,
	as the probabilities of potentially compatible relatives surviving are less
	severely bimodally distributed (Figure \ref{mammal_hist} and 
	\ref{bird_hist} F), and thus have more of a  non-binary effect
	upon the Priority Scores.
	% talk about pessimistic model being non-realistic / how notable is result
	
	Realistically being able to estimate the phylogenetic distance at which
	an interspecies restoration would be viable will be difficult. It will
	require a large number of successful interspecies restorations to
	have been performed, particularly as the viability threshold
	is likely to vary widely
	between and within taxa. Currently, the best working example of interspecies
	restoration is that of the black-footed ferret, whose donor species, the
	domestic ferret, is separated by around 1.5 million years of evolution
	\citep{frittsConservationFirstCloned2022}. Studies performed on xenomitochondrial
	cybrids (cells where the native mtDNA has
	been destroyed, and mtDNA from a cell of a different species is introduced
	by fusing an enucleated cell) also inform where potential thresholds
	may lie.
	For example, human nuclear DNA
	is compatible with gorilla and chimpanzee mtDNA (MRCA 8mya and 6mya respectively)
	but not orangutan mtDNA (MRCA 14mya)
	\citep{kenyonExpandingFunctionalHuman1997}. However, human-gorilla and
	human-chimp xenomitochondrial cybrids did have reduced
	OXPHOS Complex I activity by 40\%
	\citep{barrientosHumanXenomitochondrialCybrids1998}, which may have potential effects on reproductive success. Humans carrying mutations with similar effects on Complex
	I activity were developmentally arrested and usually died before 2 years of
	age \citep{gershoniMitochondrialBioenergeticsMajor2009}. It is also possible
	that compatibility between species is not bi-directional; it was possible to
	maintain human mtDNA in an orangutan nuclear background, despite the opposite
	case not being possible \citep{bayona-bafaluyFastAdaptiveCoevolution2005}.
	
	I
	found that
	larger values of $T$ increase the number of likely candidate species
	for interspecies restoration. This is
	expected, as larger values of $T$ increase the number of potentially
	compatible species. The greater relative increase in the number of high priority
	species for more pessimistic, or longer timeframe, models of extinction
	probabilities, is due to the higher probabilities of extinction for non-threatened
	species in these models. This reduces the probability of a potentially
	compatible relative surviving.
	An example is if some Critically Endangered species had 2 near relatives,
	both Vulnerable, the Priority Scores
	under each model would be 0.97, 0.85 and 0.36 for the 50 year, 500 year and
	pessimistic models respectively (calculated from Equation \ref{equation}).
	These results
	show that expanding
	the capability of interspecies cloning approaches to greater phylogenetic
	distances would vastly increase the number of species for which this would be
	a viable conservation approach, particularly in scenarios of continually
	accelerating rates of extinction, or over particularly long time frames.
	However, it also emphasises that there are some particularly
	valuable species, for which it is probable that related species that
	have very recently diverged survive. We should ensure that we have
	appropriately and safely biobanked high-quality samples of these species,
	and make efforts to create stable cell lines derived from them.
	
	Prioritising species for biobanking yielded clearly
	superior results for the number of potential restorations enabled
	over randomly choosing species. The slightly improved
	success of a random method in mammals when compared to birds can be
	explained as mammals have a greater proportion of threatened
	species (11.2\% and 6.2\% respectively are Critically Endangered
	or Endangered). The most interesting and perhaps surprising result is
	that a category-based optimisation strategy appeared to be superior to
	one based on Priority Scores. This may be due to the choice of $T$ for
	this set of simulations. Most species will have
	some potentially compatible donor survive within a $T$ of 10mya, and we
	can see from Figure \ref{thresh_viz} that almost all Critically
	Endangered species are considered high priority at this level. I
	believe that rerunning these simulations using a lower $T$ of
	3mya or less would show a Priority Score based optimisation
	outperforming category based optimisation. At this level, many
	Critically Endangered species will have no potentially compatible
	relative species. Therefore, it may be more effective to biobank
	species that have a medium risk of extinction themselves, but for
	whom they have a high probability of survival of a potentially
	compatible donor species.
	It is also more likely that $T$ values of around this level would
	be more realistic upper bounds for compatibility for interspecies
	cloning. As explored above, successful interspecies cloning
	has only been performed with species separated by less than 1.5
	million years of evolution \cite{sandlerEthicalAnalysisCloning2021}.
	A fruitful avenue for
	further research would be to compare these optimisation metrics
	at a range of $T$ values to see if the relationship holds.
	
	Optimising conservation spending for a metric of choice
	(in this case simply number of extinctions)
	has a large effect on the cost-effectiveness.
	The curved relationship between spending and number of extinctions for the
	optimised approach is a function of the order in which species are selected.
	Species of greatest cost effectiveness are selected first,
	making the slope steep, and as less cost-effective species are continually
	selected, the slope of the curve lessens. The random approach
	has a linear relationship, as species are sampled equally
	from across the cost effectiveness distribution for any given level of spending.
	Even the optimised approach requires a high level of spending,
	\$90 million over 10 years, to have a noticeable
	effect on the number of extinctions compared to the baseline stochastic variation.
	This level of spending becomes even more significant when considered that
	this data is for only one class, birds, and other classes would likely require
	a similar outlay. There are possible metrics by which this
	could be an overestimate of the actual spending required, as some
	items such as habitat restoration could affect the prospects
	of more than one species
	\citep{mccarthyFinancialCostsMeeting2012}. However, it is very unlikely that
	conservation spending would be optimised in this way in the real world, as
	conservation spending is rarely allocated on a cost-for-value basis
	based solely on extinction risk. Thus, the \$90 million estimate
	seems like a good lower bound for the real-world spending required
	to noticeably affect extinctions.
	The amount of spending required for biobanking compares
	favourably with these estimates. Based on informal conversations with staff at the
	Molecular Collections Facility at the Natural History Museum, material costs
	of biobanking are low, such that even storing a sufficient sample of all bird
	species (usually considered to be 10 samples each for 50 individuals
	\citep{harwoodDevelopingImplementingPrioritisation2021}) would be
	a fraction of the cost of the required conservation spending to reduce
	extinctions. The more significant portion of the costs of cloning-based
	restoration approaches would lie in sample collection and costs at the
	stage of cloning. Since many studies are already taking samples of our
	biodiversity for their own purposes, this presents an opportunity to ensure
	that the unused samples are stored with an eye to long-term
	preservation.
	
	Attempts to draw generalisable conclusions from this study are limited by a
	number of factors. One of these is that, as in the case of the black-footed
	ferret, interspecies restoration is likely to be most practical and
	effective when used with populations that are close to, but have not yet
	undergone extinction. Cross-species cloning and other methods can then be
	used to reintroduce genetic diversity to a population which has entered into
	the extinction vortex, where inbreeding causes rapid loss of diversity and
	fixation of deleterious traits. It would have been difficult to create a
	Priority Score for species based on their entry into an extinction vortex, as
	it is currently difficult to predict when species
	will enter an extinction vortex. There are only a few examples of when an
	extinction vortex has been predicted or declared
	in extant populations \citep{williamsextinctionvortex2020}, and quantitative
	models for prediction have only recently been developed
	\citep{nabutanyiModelsEcoEvolutionaryExtinction2021}. Therefore, I think that
	an analysis which assumes material will be used for interspecies
	cloning after extinction
	is well-supported, and
	this can also serve as a proxy for a species entering the extinction
	vortex.
	
	Another potential source of bias in the Priority Scores regards defining and
	classifying what constitutes a species versus a subspecies, and to which
	taxonomic level analyses should be performed.
	Species classification is a difficult and thorny subject, and reclassification
	of groups is a constantly occurring process
	\citep{sangsterApplicationSpeciesCriteria2014}.
	Classifying a population as two distinct species as compared to one could have
	a large impact on its Priority Score. For example, if some Endangered species
	had no compatible relatives, it would result in a Priority Score of 0.
	However, if this species was split into two separate species, both with
	Endangered classifications, their Priority Scores would both be 0.24.
	Artificially inflating the
	number of closely related taxa will bias the Priority Scores via this
	process.
	
	Furthermore, this analysis assumes that the use of samples for species
	restoration is the primary aim of any biobanked material, unlike other
	methods for prioritisation such as
	\citet{harwoodDevelopingImplementingPrioritisation2021}
	and \citet{mooneyValueExSitu2021}.
	This is a limited perspective on the use of biobanked samples, as they may be
	useful for basic scientific research, for other conservation approaches, or
	for scientific applications which are invented in the future. This
	analysis also fails to take into account aspects such as the current number
	of stored samples for a given species, which would be important for
	prioritisation as we can then focus our efforts on under-represented species.
	The insights gained from this more focused analysis can however be used
	to inform prioritisation efforts when considered alongside other important
	aspects such as those discussed above.
	
	
	
	There are also ethical considerations regarding prioritisation and interspecies
	restoration methods.
	Any attempt to create metrics to prioritise choice of species, even in areas
	where this is well-established, such as conservation prioritisation via
	EDGE scores \citep{isaacMammalsEDGEConservation2007} and other
	similar methods, can be controversial.
	It seems distasteful to have to choose between saving different
	species \citep{bottrillConservationTriageJust2008},
	and also because there are so many potential measures of a species' value,
	such as cultural worth, ecosystem contribution or evolutionary diversity
	\citep{breithoffArkBankExtinction2020}.
	Cross-species restoration methods are also controversial in and of themselves,
	as it can be argued that you are not truly recreating the target species.
	The mitochondria will be descended from the donated enucleated oocytes, and
	the interaction between the mitochondria and the products of the nuclear
	DNA may affect the phenotype
	\citep{shapiroPathwaysDeextinctionHow2017}. The black-footed ferret restoration
	project plans to breed the domestic ferret mtDNA from clonal lineages of
	the black-footed ferret, circumventing this issue
	\citep{sandlerEthicalAnalysisCloning2021}. However, differences in the
	gene-environment interactions for restored individuals, such as differing
	maternal factors and womb conditions, diet, and social structures, may
	be even more significant \citep{shapiroPathwaysDeextinctionHow2017}.
	This raises questions about whether we should
	be using our resources to create such genetic hybrids, as opposed to spending
	these resources on more standard conservation methods.
	Alternatively, it may be more accurate to think of restored species
	as ecological proxies of the ancestral species, which are functionally
	equivalent and occupy the same niche. This is the approach
	the IUCN has taken in the IUCN SSC Guiding Principles on
	Creating Proxies of Extinct Species
	for Conservation Benefit \citep{iucn2016iucn}. Considering restored species
	as ecological proxies suggests another method for prioritisation, based upon
	their ecological importance. It may be worthwhile to develop a priority metric
	in the future based upon the keystone species concept
	\citep{davicLinkingKeystoneSpecies2003}, or another similar method of assessing
	ecological importance.
	
	Acknowledging the difficulty of resolving such ethical and moral considerations,
	and of the scientific difficulty of expanding interspecies cloning approaches
	beyond very close relatives, we are presented with a rapidly
	closing window to collect samples of many of these endangered species before
	they unfortunately become extinct. This seems like a low-risk venture, as the
	material cost of sample storage is relatively inexpensive, and
	need not divert resources from other spending on conservation.
	These samples would
	also be useful for many other applications, even if interspecies restoration
	proves unviable or is ruled out for ethical reasons. My results
	in the conservation spending and potential restoration simulations show that
	large increases in the efficacy of conservation efforts can be made, if
	criteria for optimisation can be agreed and implemented.
	Spending on biobanking and cloning-based restoration methods should
	be thought of as a complement to, rather than a replacement for,
	traditional conservation spending. It may even allow current conservation
	spending to have greater leverage, by improving aspects of population health
	such as genetic diversity.
	Finally, greater coordination among scientists and existing biobanks
	can improve their contribution, as we can ensure that the samples existing in
	biobanks reflect the diversity of life, and that samples taken are stored with
	a view to the long-term, to avoid losing these precious resources.
	
	\pagebreak
	
	\section{Acknowledgments}
	First and foremost, I would like to thank my primary supervisor, Dr. James
	Rosindell, for all of his help and support. I would also like to thank
	Prof. Francisco Pelegri at the University of Wisconsin Madison for very
	interesting discussions and for generating the initial idea for this
	project with James. Thanks to everyone at the Molecular Collections
	Facility at the Natural History Museum, and from the Frozen Arks and
	Cryo Arks teams, for their help and discussions. Particularly, thanks to
	Jacqueline Mackenzie-Dobbs for giving James and I a tour of the MCF
	and providing
	insights into the day-to-day running of a biodiversity biobank.
	
	Alongside this, I would like to thank all of my friends and family
	for their support and for making the whole process enjoyable. Particularly,
	I'd like to thank David, Lizzie and Liam for helping each other get
	through the last few days of writeup together. Thanks also to Kate
	and to Mam for reading my (quite rough at the time) draft and providing
	lots of useful feedback.
	
	I also acknowledge computational resources and support provided by the Imperial College Research Computing Service (http://doi.org/10.14469/hpc/2232).
	
	\pagebreak
	
	\section{Data and Code Availability Statement}\label{availability_statement}
	Both the raw and cleaned data used for the project is available to
	access from OneDrive (https://imperiallondon-my.sharepoint.com/:f:/g/personal/etm21\_ic\_ac\_uk/EofukRneNttIqujmanNQ-ekBsrHLURwudxs50NpP8wbm-g?e=LD6jK1).
	
	Code is available to access on GitHub (https://github.com/eamonnmurphy/biobanking\_prioritisation).
	
	%\section{Appendix}
	%\subsection{Bird Downlisting Conservation Expenditure Estimates}\label{expen}
	%\begin{table}[h]
	%	\csvreader[longtable=|c|c|c|,
	%		table head=\hline Species & Status & Required Expenditure (US \$)\\\hline,
	%		late after line=\\]%
	%	{cleanedexpen.csv}{Species=\species,Status=\status,Required expen=\required}%
	%	{\species & \status & \required}%
	%\end{table}

	\pagebreak
	
	\bibliographystyle{apalike}
	\bibliography{bibtex,manual}
\end{document}