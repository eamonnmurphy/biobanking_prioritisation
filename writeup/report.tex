\documentclass[10pt]{article}
\usepackage{csvsimple}
\usepackage{amsmath}
\usepackage{a4wide}

\title{Prioritising species sampling for genetic biobanking}
\author{Eamonn Murphy}

\begin{document}
	\maketitle
	
	\pagebreak
	
	\section{Introduction}

	Scientists believe that we may be living through the time of the Sixth 
	Great Mass Extinction \cite{barnoskyHasEarthSixth2011}. Anthropogenic changes to the natural environment, such 
	as climate change and pollution, have already driven many species extinct and 
	threaten to cause the extinction of many more in the coming centuries \cite{ceballosVertebratesBrinkIndicators2020}. 
	Traditional conservation strategies focus on mitigating the impacts of human
	activites, as well protecting some of our most vulnerable and 
	threatened species. However, this may not be sufficient to protect some of the most
	vulnerable species, as these approaches are often costly to implement, and 
	the political will can be lacking. Another approach, previously confined to
	science fiction stories, but recently made possible by advances in 
	biotechnology, is to clone individuals from threatened (and potentially even
	extinct) species using stored biological material \cite{loiGeneticRescueEndangered2001}. In order to make this approach possible, we must biobank samples from threatened species now, before it is too
	late.
	
	Traditional conservation methods include habitat
	protection and restoration, as well as protecting species from poaching and other
	harmful human activities \cite{mccarthyFinancialCostsMeeting2012}. Breeding programmes have also played an important role,
	whether in zoos, aquariums, wildlife preserves or other venues. The goal of breeding
	programmes, has been to increase the number of individuals of the species, while
	maintaining its genetic diversity as best as possible. However, this can be 
	difficult, particularly when the source wild population is small, or if the species has
	difficulties with breeding in captivity (which often mean most of the successful
	breeding is done with a few individuals). It has been common to collect samples
	from these captive individuals, such as frozen cells, DNA and stem cells \cite{angelesChallengesDevelopmentBiodiversity2022, breithoffArkBankExtinction2020}. Now, it
	should be possible to use these stored samples to assist in the goal of maintaining
	genetic diversity within captive, and ultimately wild, populations.
	
	It has been possible for a few decades now to create clones of existing individuals \cite{campbellSheepClonedNuclear1996}.
	This approach has traditionally relied upon the use of egg cells from the donor
	individual, which can be difficult and expensive to extract, particularly if you are
	working with an endangered species, where every individual is particularly valuable.
	The method for doing this is somatic cell nuclear transfer, which involves replacing
	the nucleus of the egg cell using a nucleus taken from a somatic cell. The somatic
	nucleus is then reprogrammed by the egg cytoplasmic factors to become a zygotic nucleus,
	and the egg will begin to multiply, forming a blastocyst. This blastocyst can then be
	implanted into a host embryo, in which it will grow and develop into a new, cloned
	individual. Under recent developments, it is now possible to use a cross-species
	approach for this process, where a somatic nucleus from the target species is
	implanted into an egg cell obtained from some closely related species \cite{loiGeneticRescueEndangered2001}.
	
	For example, this approach has recently been used to clone an individual of the black-footed ferret \cite{ConservationFirstCloned, wiselyRoadMap21st2015, sandlerEthicalAnalysisCloning2021}. The black-footed ferret is an endangered species native to the 
	USA, which had been thought extinct in the late 1970s. In 1987, cell lines from a few
	individuals of the species were preserved via deep-freezing. Since then, there has been
	a breeding programme that has attempted to increase the numbers of the species, in order
	to complement the wild population. However, despite the best efforts of the breeding
	programme, genetic diversity within the population has been continually declining. For
	this reason, it was decided to attempt to clone an individual using the stored cell
	lines, by implanting the somatic nucleus into an egg cell from the domestic ferret, 
	and developing the embryo in a surrogate domestic ferret mother. The stored lines contain
	a lot of genetic diversity that has been lost in the current population, and it is hoped
	to use the successfully cloned individual in the breeding programme to restore some of
	this genetic diversity to the population \cite{ConservationFirstCloned}.
	
	Cross-species cloning may be an effective method for certain taxa, such as mammals.
	However, in other taxa such as birds, cloning is not possible \cite{ConservationFirstCloned}. Alternative methods
	have been developed which may be able to perform a similar role. It is possible to 
	introduce primordial germ cells from a target species into some host species, for
	example the chicken. These germ cells will then migrate to the gonads of the host
	species, where they become sex cells and begin to produce gametic cells.
	
	Maintaining genetic diversity within a population is key to its long-term health and
	survival. Existing genetic diversity is important to allow for selection under
	evolutionary pressures \cite{harmonConservationSmallPopulations2010}. This allows a population to adapt to changes in its environment,
	and new threats such as pathogens or climatic changes \cite{mccallumTasmanianDevilFacial2008}. For example, having a variety
	of variants of antigenic receptors within a population will likely mean that a pathogen
	binds more closely to cells in some individuals than others, thus being less pathogenic
	to some and allowing them to survive and reproduce. Genetic rescue is another phenomenon
	that relies on genetic diversity within the population, whereby selection on existing
	genetic variation can give rise to traits which "rescue" a population from new external
	conditions.
	
	An important concept for understanding the genetic diversity within a population is the
	effective population size ($N_e$). The effective population size is a measure of
	population size which also takes into account the relatedness between individuals in
	a population, thus serving as a good measure of the total genetic diversity. We can
	create a threshold using the effective population size, below which populations enter
	what is known as the extinction vortex \cite{harmonConservationSmallPopulations2010}. This threshold is usually set at an $N_e$ of
	around 50. Upon entering the extinction vortex, populations not only struggle to
	adapt to new environmental changes, but begin to suffer from the ill effects of
	inbreeding, which can lead to decreased fitness and reduced fertility. It is very
	difficult to rescue a population once it has entered the extinction vortex, as genetic
	drift will tend to overpower mutation at such small population sizes and genetic
	variation will continually be lost \cite{harmonConservationSmallPopulations2010}. One of the promising aspects of cloning techniques 
	for conservation is that it will allow conservationists to restore genetic diversity to
	populations which have entered an extinction vortex, as in the case of the black-footed
	ferret above \cite{wiselyRoadMap21st2015}. 

	
	Besides rescuing or restoring genetic diversity to highly vulnerable populations,
	it seems likely that we will be able to extend this approach to clone individuals from species which
	have recently become extinct. In fact, there is an ongoing project to restore
	the Wooly Mammoth by a similar approach, which is a much more difficult project
	as it necessitates working with ancient DNA, meaning low coverage reads and 
	missing chunks of the genome \cite{novakDeExtinction2018}. By comparison, restoring species from DNA stored 
	in modern cryopreservation facilities should be relatively doable.
	
	Often, conservation resources are focused on species with high visibility or
	high cultural appeal \cite{gunnthorsdottirPhysicalAttractivenessAnimal2001}. Although this approach can help to drive resources
	towards conservation, it does not optimise the use of the resources we have, 
	and can result in oversights which leave behind species of high worth on other
	metrics, such as ecological importance or phylogenetic diversity. This has been
	the basis of approaches such as EDGE scores, which prioritise conservation 
	resources based on aims such as maximising retained phylogenetic diversity \cite{isaacMammalsEDGEConservation2007}. In 
	this case, we aim to prioritise species for biobanking based solely on the 
	likelihood of the material becoming useful. Currently, biobanks such as 
	Nature's Safe simply aim to biobank whatever material is available. While this %need citation
	can be a cost effective approach, and avoids potentially controversial 
	decisions comparing the worth of various species, it means that the samples are
	biased, particularly towards species housed in zoos \cite{mooneyValueExSitu2021}. Zoos are generally more
	likely to contain larger mammals and certain taxa like primates, and as such
	give an unrepresentative sample of the total biological diversity of animals \cite{gunnthorsdottirPhysicalAttractivenessAnimal2001}.
	
	Under a scenario of limited resources, a key consideration for using cloning and other
	biotechnological techniques to restore species and population genetic diversity, is
	which species to biobank first \cite{harwoodDevelopingImplementingPrioritisation}. We can build a metric, the Priority Score, that approaches this 
	question by utilising phylogenetics, and data on conservation status. The cross-species approach for species restoration appears most promising at present, as it avoids using
	individuals of a valuable threatened species for risky procedures such as egg cell
	extraction and blastocyst implantation, and may also allow us to leverage existing
	knowledge on best breeding practices within the donor species \cite{wiselyRoadMap21st2015}. In order for a
	cross-species approach to be possible, we need biobanked material from the target
	species, as well as the presence of some sufficiently closely related species to act
	as a donor of egg cells and a host for embryos. Therefore, the highest priority species
	for biobanking are those which have a high probability of extinction, alongside a high
	probability of the survival of a closely related species, within some threshold. The current best test of
	relatedness is phylogenetic distance, particularly in a general, cross-taxa approach, so we can calculate these prioritisation
	scores by combining IUCN Red List data with phylogenetic trees of the taxa of 
	interest.
	
	As well as generating a set of priority scores for biobanking, based on the
	outlined approach, we also want to get a sense of the usefulness and
	robustness of this approach. One way to do this is to simulate a set of future
	scenarios and compare them to each other. These can compare the number of species
	preserved when biobanking is based upon the priority list, as well as a random
	approach, and the scenario where cloning-based restoration of species is not
	performed at all. We can also get some sense of the financial cost-effectiveness of
	this endeavour when compared to traditional conservation approaches. Estimates of the
	cost of downlisting (moving from one IUCN threat status to the next, e.g. Critically
	Endangered to Endangered) around 200 different bird species were made by
	species experts \cite{mccarthyFinancialCostsMeeting2012}, and we can use these costs as a base comparison to see e.g. what
	amount of phylogenetic diversity would be saved by each approach, and at what
	cost level would cloning-based approaches need to be viable to be competitive
	with traditional conservation approaches \cite{nunesPriceConservingAvian2015}. Of course, the suggestion is not to replace
	existing conservation spending with spending on biobanking and cloning; it would
	be to supplement the existing conservation efforts with this technique, and the
	financial analysis simply gives a good baseline on which to analyse it.

	%still need to discuss - 
	% 	where to set the thresholds / how are these set
	%	the use of conversions from IUCN status to extinction probabilities
	
	
	
	\section{Methods}
	\subsection{Data Sources}
	Conservation status for mammals and birds was sourced from the IUCN Red List 2020 \cite{iucnIUCNRedList2021, IUCNRedList2012}. This was combined with 100 phylogenetic trees for each class, kindly provided by Dr. Rikki Gumbs (EDGE). The conservation statuses were converted to likelihoods of extinction, according to the transformations in Table \ref{ext_prob}.
	
	\begin{table}[b]
		\begin{tabular}{|c|c c c c|}
			\hline
			Status & Abbreviation & IUCN 50 years & IUCN 500 years & Pessimistic \\
			\hline
			Least Concern & LC & 0.00005 & 0.0004 & 0.2 \\
			Near Threatened & NT & 0.004 & 0.02 & 0.4 \\
			Not Available & NA & 0.004 & 0.02 & 0.4 \\
			Data Deficient & DD & 0.004 & 0.02 & 0.4 \\
			Not Evaluated & NE & 0.004 & 0.02 & 0.4 \\
			Vulnerable & VU & 0.05 & 0.39 & 0.8 \\
			Endangered & EN & 0.42 & 0.996 & 0.9 \\
			Critically Endangered & CR & 0.97 & 1 & 0.99 \\
			Extinct in the Wild & EW & 1 & 1 & 1 \\
			\hline
		\end{tabular}
		\caption{Conversion of IUCN threat status to probabilities of extinction, under three
			different scenarios \cite{mooersConvertingEndangeredSpecies2008}}\label{ext_prob}
	\end{table}

	Data was also obtained on the estimated cost of conservation spending needed to downlist 206 different bird species, from \cite[McCarthy \textit{et al.} 2012]{mccarthyFinancialCostsMeeting2012}. Downlisting means moving a species from one IUCN threat category to the one below it, e.g. Critically Endangered to Endangered. This data was collected via expert surveys, and was formatted as the estimated spending per year in US \$ needed over 10 years to result in species downlisting. Amounts were converted using an inflation estimator to 2022 US \$. Mismatches between the species names in this data and the 2020 IUCN data were resolved using Global Names Verifier \cite{}. Any species without a clear synonymic or misspelling match were removed from the data. Species who had changed IUCN threat status during the intervening period were also removed from the data. This left a total of 166 species to use for the analysis. The cleaned data is available to view in the appendix.
	
	Data on estimated material costs of sample storage in a biobank was obtained from Jacqueline Mackenzie-Dobbs, manager of the Molecular Collections Facility at the Natural History Museum. These estimates include consumable purchase, liquid nitrogen / electricity usage, service and maintenance costs and equipment depreciation. However, it does not include a large portion of the costs, such as staff time, insurance, building space and other overheads.
	
	\begin{table}
		\begin{tabular}{|c|c|c|c|}
			\hline
			Cryovial Volume (mL) & Storage Temp ($^\circ C$) & Cost per Sample Year 1 & Annual Cost Ongoing \\
			\hline
			2.0 & -20 & £1.08 & £0.05 \\
			 & -80 & £1.18 & £0.15 \\
			 & -196 & £1.25 & £0.22 \\
			\hline
			0.5 & -20 & £0.27 & £0.01 \\
			 & -80 & £0.29 & £0.04 \\
			 & -196 & £0.31 & £0.06 \\
			\hline
		\end{tabular}
		\caption{Estimated material costs for sample storage in the NHM Molecular Collections Facility}
	\end{table}

	\subsection{Priority Score Calculation}\label{ps_method}
	Priority scores were calculated for each species in the two taxa of interest, birds and mammals. The formula for calculating this score is outlined below, where the probability of extinction
	for the species of interest is multiplied by the chance of any closely related species surviving
	($RS$).
	A closely related species is defined by being within the threshold $T$, which is the time in 
	millions of years to the most recent common ancestor (MRCA) of the two species. If there are no
	relatives within $T$ million years, the probability of survival for a closely related species is
	0, and thus the Priority Score is 0. For any species with at least one related species within $T$ 
	million years, $RS$ will be 1 minus the product of the probabilities of extinction of the n
	species within $T$ million years, i.e. the probability of any of these related species
	surviving.
	
	\begin{align}
		Priority\ Score &= Pr(Extinction) * [1 - \prod_{i}^{n}{Pr(Extinction\ for\ related\ species\ i)}] \\
		PS &= P_e * RS \\
		RS &= \begin{cases}
			0 & \text{If there are 0 species within $T$ million years of s} \\
			1 - \prod_{i}^{n}{P_e} & \text{Otherwise}
		\end{cases}
	\end{align}

	In order to deduce the number of closely related species for each species, phylogenetic trees were used. 100 different phylogenetic trees were used for each taxa, in order to avoid biasing the results to any particular phylogeny. The priority scores for each species were calculated for each phylogeny, and the mean across all 100 trees was then calculated and used as the final
	priority score for further steps. A number of different models for the likelihood of species
	extinction are available, as shown in Table \ref{ext_prob}. Each of these models was used to
	calculate a different set of priority scores, changing the probabilities of extinction as
	appropriate for each model.
	
	\section{Results}
	\subsection{Prioritising Species for Biobanking}
	The primary aim of this project was to explore methods for prioritising sample collection and
	storage for genetic biobanking, when the aim of the sample storage is for use in cross-species
	restoration techniques. These priority scores were calculated for two taxa of broad interest,
	mammals and birds. Mammals were of interest as attempts have already been made to use
	cross-species methods to assist in conservation within this taxa, as in the example of the
	black-footed ferret. Birds were used as a comparison taxa, and also because there are data
	available on estimated costs of conservation for bird species, which will be used in a later section. The priority scores using the methods laid out in Section \ref{ps_method}
	
	\section{Discussion}
	Another potential source of bias in our estimates lies within species
	classification in the phylogenetic trees. There will always be debate whether
	to classify closely related taxa as seperate species or only subspecies. This
	can create a source of bias, particularly if there is a tendency within certain
	genera or orders to classify as species vs. subspecies and vice-versa. Since it
	is necessary for closely related species to survive, artificially inflating the
	number of closely related taxa will bias the priority scores. However, this
	effect should be evened somewhat by the fact that each of these species will 
	have their own conservation status, which will be incorporated into the
	calculations. A sensitivity analysis needs to be carried out to get a better
	sense of the size of this effect.
	
	\section{Appendix}
	\subsection{Bird Downlisting Conservation Expenditure Estimates}
	\csvreader[tabular=|c|c|c|,
		table head=\hline Species & Status & Required Expenditure\\\hline,
		late after line=\\\hline]%
	{cleanedexpen.csv}{"Species"=\species,"Status"=\status,"Required_expen"=\required}%
	{\thecsvrow & \status & \required}%
	
	
	\bibliographystyle{plain}
	\bibliography{MSc_Thesis}
\end{document}