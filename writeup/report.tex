\documentclass[10pt]{article}
\usepackage{csvsimple}
\usepackage{amsmath}
\usepackage{a4wide}
\usepackage{graphicx}
\usepackage[square]{natbib}

\title{Prioritising species sampling for genetic biobanking}
\author{Eamonn Murphy}

\begin{document}
	\maketitle
	
	\pagebreak
	
	\section{Introduction}

	Scientists believe that we may be living through the time of the Sixth 
	Great Mass Extinction \citep{barnoskyHasEarthSixth2011}. Anthropogenic changes to the natural environment, such 
	as climate change and pollution, have already driven many species extinct and 
	threaten to cause the extinction of many more in the coming centuries \citep{ceballosVertebratesBrinkIndicators2020}. 
	Traditional conservation strategies focus on mitigating the impacts of human
	activities, as well protecting some of our most vulnerable and 
	threatened species. However, this may not be sufficient to protect some of the most
	vulnerable species, as these approaches are often costly to implement, and 
	the political will can be lacking. Another approach, previously confined to
	science fiction stories, but recently made possible by advances in 
	biotechnology, is to clone individuals from threatened (and potentially even
	extinct) species using stored biological material \citep{loiGeneticRescueEndangered2001}. In order to make this approach possible, we must biobank samples from threatened species now, before it is too late.
	
	Traditional conservation methods include habitat
	protection and restoration, as well as protecting species from poaching and other
	harmful human activities \citep{mccarthyFinancialCostsMeeting2012}. Breeding programmes have also played an important role,
	whether in zoos, aquariums, wildlife preserves or other venues. The goal of breeding
	programmes, has been to increase the number of individuals of the species, while
	maintaining its genetic diversity as best as possible. However, this can be 
	difficult, particularly when the source wild population is small, or if the species has
	difficulties with breeding in captivity (which often mean most of the successful
	breeding is done with a few individuals).
	
	Maintaining genetic diversity within a population is key to its long-term health and
	survival. Existing genetic diversity is important to allow for selection under
	evolutionary pressures \citep{harmonConservationSmallPopulations2010}. This allows a population to adapt to changes in its environment,
	and new threats such as pathogens or changing environments
	\citep{mccallumTasmanianDevilFacial2008}. Evolutionary rescue, where a population
	avoids extinction through adaption and selection
	\citep{bellEvolutionaryRescue2017}, relies on genetic diversity that
	exists within the population.
	An important concept for understanding the genetic diversity within a population is the
	effective population size ($N_e$). The effective population size is a measure of
	population size which also takes into account the relatedness between individuals in
	a population, thus serving as a good measure of the total genetic diversity
	\citep{frankhamEffectivePopulationSize1995}. We can
	create a threshold using the effective population size, below which populations enter
	what is known as the extinction vortex \citep{harmonConservationSmallPopulations2010}. This threshold is usually set at an $N_e$ of
	around 50. Upon entering the extinction vortex, populations not only struggle to
	adapt to new environmental changes, but begin to suffer from the ill effects of
	inbreeding, which can lead to decreased fitness and reduced fertility. It is very
	difficult to rescue a population once it has entered the extinction vortex, as genetic
	drift will tend to overpower mutation at such small population sizes and genetic
	variation will continually be lost \citep{harmonConservationSmallPopulations2010}.
	Currently existing methods to improve genetic diversity include genetic rescue,
	where new alleles are introduced to a population through immigrants
	\citep{whiteleyGeneticRescueRescue2015}. However, this method can only be used to
	improve the prospects of sub-populations of a species, and not the species as a
	whole, as there are is no traditional method for improving genetic diversity
	across an entire species.
	
	
	It has been common to collect samples
	from these captive individuals, such as frozen cells, DNA and stem cells \citep{angelesChallengesDevelopmentBiodiversity2022, breithoffArkBankExtinction2020}. Now, it
	should be possible to use these stored samples to assist in the goal of maintaining
	genetic diversity within captive, and ultimately wild, populations.
	
	It has been possible for a few decades now to create clones of existing individuals \citep{campbellSheepClonedNuclear1996}.
	This approach has traditionally relied upon the use of egg cells from the donor
	individual, which can be difficult and expensive to extract, particularly if you are
	working with an endangered species, where every individual is particularly valuable.
	The method for doing this is somatic cell nuclear transfer, which involves replacing
	the nucleus of the egg cell using a nucleus taken from a somatic cell. The somatic
	nucleus is then reprogrammed by the egg cytoplasmic factors to become a zygotic nucleus,
	and the egg will begin to multiply, forming a blastocyst. This blastocyst can then be
	implanted into a host embryo, in which it will grow and develop into a new, cloned
	individual. Under recent developments, it is now possible to use an inter-species
	approach for this process, where a somatic nucleus from the target species is
	implanted into an egg cell obtained from some closely related species \citep{loiGeneticRescueEndangered2001}.
	
	For example, this approach has recently been used to clone an individual of the black-footed ferret
	\citep{frittsConservationFirstCloned2022, wiselyRoadMap21st2015, sandlerEthicalAnalysisCloning2021}. The black-footed ferret is an endangered species native to the 
	USA, which had been thought extinct in the late 1970s. In 1987, cell lines from a few
	individuals of the species were preserved via deep-freezing. Since then, there has been
	a breeding programme that has attempted to increase the numbers of the species, in order
	to complement the wild population. However, despite the best efforts of the breeding
	programme, genetic diversity within the population has been continually declining. For
	this reason, it was decided to attempt to clone an individual using the stored cell
	lines, by implanting the somatic nucleus into an egg cell from the domestic ferret, 
	and developing the embryo in a surrogate domestic ferret mother. The stored lines contain
	a lot of genetic diversity that has been lost in the current population, and it is hoped
	to use the successfully cloned individual in the breeding programme to restore some of
	this genetic diversity to the population \citep{frittsConservationFirstCloned2022}.
	
	Cross-species cloning may be an effective method for certain taxa, such as mammals.
	However, in other taxa such as birds, cloning is not possible \citep{frittsConservationFirstCloned2022}. Alternative methods
	have been developed which may be able to perform a similar role. It is possible to 
	introduce primordial germ cells from a target species into some host species, for
	example the chicken. These germ cells will then migrate to the gonads of the host
	species, where they become sex cells and begin to produce gametic cells.
	
	As the example of the black-footed ferret shows, the inter-species restoration approach
	relies upon biobanked samples preserving the genetic diversity of the species, and
	serving as sources for the needed genetic material. This necessitates storing now,
	any samples we might wish to use in the future, particularly given the current
	rates of extinction and population decline. Under a scenario of limited resources,
	a key consideration is
	which species to biobank first \citep{harwoodDevelopingImplementingPrioritisation2021}.
	Often, conservation resources are focused on species with high attractiveness or
	cultural appeal \citep{gunnthorsdottirPhysicalAttractivenessAnimal2001}. Although this approach can help to drive resources
	towards conservation, it does not optimise the use of the resources we have, 
	and can result in oversights which leave behind species of high worth on other
	metrics, such as ecological importance or phylogenetic diversity. This has been
	the basis of approaches such as EDGE or HEDGE scores, which prioritise conservation 
	resources based on aims such as maximising retained phylogenetic diversity \citep{isaacMammalsEDGEConservation2007,steelHedgingOurBets2007}.
	Mirroring these approaches, we can attempt to prioritise species for biobanking
	by utilising data on phylogenetics and extinction risk.
	The cross-species approach for species restoration appears most promising at present, as it avoids using
	individuals of a valuable threatened species for risky procedures such as egg cell
	extraction and blastocyst implantation, and may also allow us to leverage existing
	knowledge on best breeding practices within the donor species \citep{wiselyRoadMap21st2015}. In order for a
	cross-species approach to be possible, we need biobanked material from the target
	species, as well as the presence of some sufficiently closely related species to act
	as a donor of egg cells and a host for embryos. Therefore, the highest priority species
	for biobanking are those which have a high probability of extinction, alongside a high
	probability of the survival of a potentially compatible donor species. The current best test of
	relatedness is phylogenetic distance, particularly in a general, cross-taxa approach, so we can calculate these prioritisation
	scores by combining IUCN Red List data with phylogenetic trees of the taxa of 
	interest.
	
	This project thus aimed to create a metric for prioritising species for
	biobanking, based on the potential usefulness of the material for inter-species
	restoration methods. Factors such as how closely related two species need to
	be for inter-species cloning to be possible could affect the order of
	prioritisation. I aimed to assess some of these factors, as well as running
	simulations to analyse the effectiveness of the developed prioritisation
	methods when compared to other possible methods. I also wished to compare
	traditional conservation spending against biobanking, by simulating the
	effect on extinctions of various levels of conservation spending, based on
	estimates from expert surveys
	\citep{mccarthyFinancialCostsMeeting2012,nunesPriceConservingAvian2015}.

	%still need to discuss - 
	% 	where to set the thresholds / how are these set
	%	the use of conversions from IUCN status to extinction probabilities
	
	
	
	\section{Methods}
	\subsection{Data Sources}
	Extinction risk categories for mammal and bird species was sourced from the IUCN Red List 2020 \citep{iucnIUCNRedList2021, iucnIUCNRedList2012}. This was combined with 100 phylogenetic trees for each class, taken from \citet{uphamInferringMammalTree2019}
	for mammals, and \citet{jetzGlobalDiversityBirds2012} respectively.
	 The extinction risk categories were converted to likelihoods of extinction, according to the transformations given by \citep{mooersConvertingEndangeredSpecies2008} (see Table \ref{ext_prob}).
	
	\begin{table}[b]
		\begin{tabular}{|c|c c c c|}
			\hline
			Status & Abbreviation & IUCN 50 years & IUCN 500 years & Pessimistic \\
			\hline
			Least Concern & LC & 0.00005 & 0.0004 & 0.2 \\
			Near Threatened & NT & 0.004 & 0.02 & 0.4 \\
			Not Available & NA & 0.004 & 0.02 & 0.4 \\
			Data Deficient & DD & 0.004 & 0.02 & 0.4 \\
			Not Evaluated & NE & 0.004 & 0.02 & 0.4 \\
			Vulnerable & VU & 0.05 & 0.39 & 0.8 \\
			Endangered & EN & 0.42 & 0.996 & 0.9 \\
			Critically Endangered & CR & 0.97 & 1 & 0.99 \\
			Extinct in the Wild & EW & 1 & 1 & 1 \\
			\hline
		\end{tabular}
		\caption{Conversion of IUCN threat status to probabilities of extinction, under three
			different scenarios \citep{mooersConvertingEndangeredSpecies2008}}\label{ext_prob}
	\end{table}

	Data was also obtained on the estimated conservation spending needed for 206 different bird species, from \citet{mccarthyFinancialCostsMeeting2012}. This data was collected via expert surveys, and was their best estimate of the cost to downlist each birds species,
	which means to move it from one IUCN risk category to the next, e.g from Critically Endangered
	to Endangered. It was formatted as the estimated spending per year in US \$ needed over 10 years to result in species downlisting. To account for inflation, amounts were converted from January 2012
	US \$ to May 2022 US \$ using the US Bureau of Labour Statistics CPI Inflation Calculator
	\citep{CPIInflationCalculator} % need to citep the website from Zotero, https://www.bls.gov/data/inflation_calculator.htm
	. Mismatches between the species names in this data and the 2020 IUCN data were resolved using Global Names Verifier \citep{mozzherinGnamesGnverifierV12022}. Any species without a clear synonymic or misspelling match were removed from the data. Species that had changed IUCN extinction risk category
	during the intervening period were also removed from the data. This left a total of 166 species to use for the analysis, and the cleaned data used is available from Appendix \ref{expen}.
	
	%\begin{table}
	%	\begin{tabular}{|c|c|c|c|}
	%		\hline
	%		Cryovial Volume (mL) & Storage Temp ($^\circ C$) & Cost per Sample Year 1 & Annual Cost Ongoing \\
	%		\hline
	%		2.0 & -20 & £1.08 & £0.05 \\
	%		 & -80 & £1.18 & £0.15 \\
	%		 & -196 & £1.25 & £0.22 \\
	%		\hline
	%		0.5 & -20 & £0.27 & £0.01 \\
	%		 & -80 & £0.29 & £0.04 \\
	%		 & -196 & £0.31 & £0.06 \\
	%		\hline
	%	\end{tabular}
	%	\caption{Estimated material costs for sample storage in the NHM Molecular Collections Facility}
	%\end{table}

	\subsection{Priority Score Calculation}\label{ps_method}
	In order to generate a descending order list for prioritising species to biobank,
	a Priority Score was calculated for each species. This Priority Score is the probability
	that the biobanked material will be useful for cross-species restoration approaches, i.e.
	that the species itself is extinct, and that it has some remaining close relative. We can
	define the set of sufficiently close relatives, $A_i(T)$, as the set of species
	 whose most recent
	common ancestor (MRCA) with the species of interest, $i$, is within $T$ million years (see
	Equation \ref{set}). We can then define the Priority Score as $P_i(T)$, where it is equal
	to the probability of extinction of the species of interest ($P_{ext}(i)$) multiplied by
	the probability of any closely related species surviving (see Equation \ref{equation}). 
	This can be calculated as 1 minus
	the probability of all closely related species becoming extinct,
	which is product of the probabilities
	of extinction of each closely related species $a$ from the previously defined set $A_i(T)$.
	This definition also holds for the case where there are no closely related species within
	$T$ mya, as by definition, the product of an empty set is 1, producing a Priority Score of 0.
	
	\begin{align}
		P_i(T) &= P_{ext}(i) * [1 - \prod_{a \in A_i(T)} P_{ext}(a)] \label{equation}\\
		A_i(T) &= \{j \in S | MRCA(j, i) \le T\} \label{set}
	\end{align}

	These priority scores were calculated for two taxa of broad interest,
	mammals and birds. Mammals were of interest as attempts have already been made to use
	cross-species methods to assist in conservation within this taxa, as in the example of the
	black-footed ferret. Birds were used as a comparison taxa, and also because there are data
	available on estimated costs of conservation for bird species. In order to produce the set $A_i(T)$
	of closely related species for each species, time since divergence was calculated using phylogenetic
	trees. 100 different phylogenetic trees were used for each taxa, to avoid biasing the results to any
	particular phylogeny. %citep source of trees from Rikki
	The priority scores for each species were calculated for each phylogeny,
	and the mean across all 100 trees was then calculated and used as the final
	priority score. A number of different models for the probabilities of species
	extinction, based on the IUCN Risk Categories, are available, as shown in Table \ref{ext_prob}.
	Each of these models was used to
	calculate a different set of priority scores, changing the probabilities of extinction as
	appropriate for each model. To explore the effect of varying $T$ on the distribution of Priority
	Scores, I calculated the Priority Score for a range of values of $T$, from 1
	to 20 million years since the MRCA. To visualise these distributions effectively,
	the number of high priority species for each calculation was counted. High priority
	species were defined as those with a Priority Score of > 0.95 (meaning a 95\%
	chance of being useful for cross-species restoration under these assumptions).
	This threshold was chosen as it is close to the extinction probabilities
	of the Critically Endangered species for the 50 year and pessimistic scenarios,
	and the extinction probabilities of the Endangered species for the 500 year
	scenario (see Table \ref{ext_prob})).
	
	\subsection{Extinction Simulations}\label{sim_methods}
	Simulated extinctions were carried out for two reasons, to deduce the usefulness of the prioritisation
	method as compared to random sampling, and to generate cost estimates of saving species from
	extinction using traditional conservation techniques. For all simulations, the IUCN 50
	year model of extinction probabilities \citep{mooersConvertingEndangeredSpecies2008} was used, alongside
	a $T$ of 10 million years. Simulations were performed by randomly drawing a number from the 
	uniform distribution between 0 and 1 for each species,
	and then comparing this to their extinction probability. If the randomly drawn number 
	was less than the extinction probability, that species was deemed to have gone extinct.
	
	To benchmark the usefulness of the prioritisation method for biobanking, a restoration simulation
	was then performed. This involved first generating a list of biobanked species of a given length,
	either randomly selected or in descending order from Priority Scores. Then, for each biobanked
	species, I checked whether they had reached extinction in the simulation; if they had, then I 
	checked whether any of their relatives from the set $A_i(T)$ of close relatives had survived.
	If both of these conditions were met, the species was considered to be capable of being "restored",
	i.e. using cross-species restoration methods for de-extinction of the species.
	
	Simulations were also used to estimate the number of extinctions prevented, for a certain level
	of traditional conservation spending. In this case, before the simulations were performed, a
	number of species were downlisted (moved down one IUCN threat category), and then simulations
	were performed as described with the new extinction probabilities, and the number of extinctions
	was recorded. To determine which species would be downlisted, a certain budget was set for each
	simulation. Then species were selected either by random sampling, or by an optimised method
	which selected the species which represented the best value for money in conservation spending,
	in descending order. Value for money was determined by dividing the cost of conservation
	by the difference in probability of extinction between the two threat categories, e.g. for
	a Critically Endangered species, this would be 0.55,
	calculated by 0.97 (extinction probability for Critically
	Endangered) minus 0.42 (extinction probability for Endangered). The cleaned cost estimates
	from \citep{mccarthyFinancialCostsMeeting2012} were only available for 166 threatened species.
	To extend this to the other 1325 species available in the IUCN Red List dataset I was 
	using, each species without a cost was randomly assigned a cost from one of the estimates within
	the same extinction risk category.
	
	\subsection{Code}
	Code for this project was written primarily in R version 4.2.1 \citep{rcite},
	and is available from Github at
	\textit{https://github.com/eamonnmurphy/biobanking\_prioritisation}.
	High-throughput scripts were
	run in the Imperial College London High Performance Computing cluster. Libraries used
	include ape Version 5.6-2 \citep{paradisApeEnvironmentModern2019},
	dplyr Version 1.0.9 \citep{dplyr},
	stringr Version 1.4.0 \citep{stringr}, and stringdist \citep{stringdist}.
	
	
	\section{Results}
	\subsection{Prioritising Species for Biobanking}
	The distribution of Priority Scores for all three assessed models (IUCN 50 years,
	IUCN 500 years and Pessimistic scenario) are shown in Figure \ref{ps_hist}, for
	a $T$ of 10 million years to the MRCA. These Priority Scores are in the form
	of a probability, ranging between 0 and 1. We can see that there is a peaked
	distribution for all models and taxa, with most Priority Scores falling into
	clearly separated buckets. This distribution is similar to the underlying
	shape of the distribution of probabilities of extinction (see Table \ref{ext_prob}
	for these values), which has a discrete value for each extinction risk
	category. We can see that the distribution of Priority Scores for the
	pessimistic model differs significantly from the others, both in isolation
	and in comparison to the underlying extinction risk values for this model.
	
	\begin{figure}
		\includegraphics{../results/combined_hist.pdf}
		\caption{Histograms showing the distribution of priority scores for
		mammals (A-C) and birds (D-F), generated using the following models for
		extinction probabilities: IUCN 50 year model (A, D), IUCN 500 year
		model (B, E) and Pessimistic model (C, F)}\label{ps_hist}
	\end{figure}

	\subsection{Effect of $T$ Upon Priority Scores}
	The change in the number of high priority species (Priority Score of > 0.95)
	with variation in the value of $T$, for each of the three different models, is
	shown in Figure \ref{thresh_viz}. For all models and taxa, the number of high
	priority species increases as we increase the value of $T$. This increase
	is sharpest at lower values of $T$, and as the number of high priority
	species approaches the number of Critically Endangered species (IUCN 
	50 year model and pessimistic model) or Endangered species (IUCN 500 year
	model), this increase tapers off, which reflects the total number of species
	whose extinction probability exceeds the high priority threshold for
	each model. The relative or percentage increase in the number of high
	priority species as $T$ increases
	is greater for the IUCN 500 year and pessimistic models.
	
	\begin{figure}
		\includegraphics[scale=0.7]{../results/threshold_viz.pdf}
		\caption{The number of species with a high priority score ($> 0.95$)
			for the IUCN 50 and 500 year models and the pessimistic scenario, for 
			mammals (A) and birds (B).}\label{thresh_viz}
	\end{figure}
	
	\subsection{Simulating the Usefulness of Biobanked Material}
	We compared the number of species that could potentially be restored using
	a random and optimised biobanking approach for cross-species restoration,
	visualised in Figure \ref{rest_sims}. The optimised biobanking approach
	results in a far greater number of potential species restorations
	compared to random biobanking, for both birds and mammals. An optimised
	biobanking approach would yield 412 and 420 potential restorations, for
	birds and mammals respectively, if 1000 species were biobanked, whereas
	a random approach would yield 64 and 34 potential restorations. The number
	of restorations for the optimised biobanking approach has a clearly stepped
	shape for both mammals and birds. The transitions between these steps
	correspond to the numbers of prioritised species for each extinction risk
	category in turn, i.e. we can see the transition point from where Critically
	Endangered species are being biobanked to Endangered species, and so on.
	The distributions have quite similar shapes for the two taxa, especially
	when taking into account the underlying numbers of species in each category.
	The random biobanking approach is, however, noticeably more successful in mammals
	than in birds.
	
	\begin{figure}
		\includegraphics[scale = 0.5]{../results/restoration.pdf}
		\caption{The number of potential species restorations, comparing
		a random and optimised biobanking approach for both mammals and birds.
		These simulations were performed using the IUCN 50 year model of
		extinction probabilites, and with a $T$ of 10my to the most recent
		common ancestor.}\label{rest_sims}
	\end{figure}

	
	\subsection{Conservation Costs and Extinction}
	The likely number of extinctions prevented for a certain level of
	conservation spending on birds was simulated, according to the
	methods set out in Methods Subsection \ref{sim_methods}. The optimised
	approach to conservation spending yielded clearly superior results,
	visualised by Figure \ref{expense_sims}. The spending required to
	change the number of extinctions from baseline was approximately
	66 times greater for the random approach, almost two orders of
	magnitude (\$60 billion vs. \$900 million). We can see from the
	linear plot that the number of extinctions using the random approach
	has a linear relationship to spending, up to a certain amount, whereas
	the optimised approach has a curved relationship.
	
	
	\begin{figure}
		\includegraphics[scale = 0.8]{../results/log_expense_sims.pdf}
		\caption{The number of simulated extinctions for birds in a 50 year period,
		using the IUCN 50 year model and a $T$ of 10my, vs. the }\label{expense_sims}
	\end{figure}
	
	\section{Discussion}
	To the best of my knowledge, there are no previous publications in the
	primary literature attempting to create metrics for prioritising
	biobanking. However, there have been theses written which concentrated
	on this task, and these attempts focused mainly on aspects such as
	samples currently existing and phylogenetic diversity 
	\citep{mooneyValueExSitu2021,harwoodDevelopingImplementingPrioritisation2021}.
	% check out these citations
	That means that this research is potentially
	the first to analyse sample prioritisation by
	a direct metric of the sample's potential usefulness in the future,
	namely its use for cross-species restoration. The aim of this research
	was to create a metric to prioritise species for biobanking for this
	use, and to analyse the factors which affected the order of priority,
	as well as test the applicability of this prioritisation to the
	real world.
	
	The distribution of Priority Scores looks similar to the underlying
	distribution of extinction risks for a $T$ of 10 million years, with some
	key differences. The Priority Scores for a given species can only be less
	than the underlying probability of extinction for that species, because
	in the case that the species goes extinct, the possibility of restoration
	still relies upon another factor, a potentially compatible relative
	surviving. The
	majority of species are not within the threatened categories, meaning
	that for most species with close relatives, their Priority Score under
	the 50 year and 500 year models is only
	mildly affected by the probability of a close relative surviving (as this
	will be near to 1). Most of the apparent difference in the distribution
	comes from species who have no close relative with a MRCA of $<$ 10mya.
	However, the distribution for the pessimistic model is more evenly spread,
	as the probabilities of extinction for even non-threatened species are
	non-negligible, and thus the probability of a close relative surviving
	is not as clustered around 0 and 1.
	% talk about pessimistic model being non-realistic / how notable is result
	
	Realistically being able to estimate the phylogenetic distance at which
	a cross-species restoration would be viable will be difficult. It will
	require a large number of successful cross-species restorations to
	have been performed, particularly as the viability threshold
	is likely to vary widely
	between and within taxa. Currently, the best working example of cross-species
	restoration is that of the black-footed ferret, whose donor species, the
	domestic ferret, is separated by around 1-3 million years of evolution
	\citep{frittsConservationFirstCloned2022}. Studies performed on xenomitochondrial
	cybrids, which are cells where the native mtDNA has
	been destroyed, and mtDNA from a cell of a different species is introduced
	by merging an enucleated cell, also inform where potential thresholds
	may lie.
	For example, human nuclear DNA
	is compatible with gorilla and chimpanzee mtDNA (MRCA 8mya and 6mya respectively)
	but not orangutan mtDNA (MRCA 14mya)
	\citep{kenyonExpandingFunctionalHuman1997}. However, human-gorilla and
	human-chimp xenomitochondrial cybrids did have reduced
	OXPHOS Complex I activity by 40\%
	\citep{barrientosHumanXenomitochondrialCybrids1998}, which may have potential effects on reproductive success. Humans carrying mutations with similar effects on Complex
	I activity were developmentally arrested and usually died before 2 years of
	age \citep{gershoniMitochondrialBioenergeticsMajor2009}. It is also possible
	that compatibility between species is not bi-directional; it was possible to
	maintain human mtDNA in an orangutan nuclear background, despite the opposite
	case not being possible \citep{bayona-bafaluyFastAdaptiveCoevolution2005}. I
	found that
	larger values of $T$ increase the number of likely candidate species
	for cross-species restoration. This is
	expected, as larger values of $T$ increase the number of potentially
	compatible species. The greater relative increase in the number of high priority
	species for more pessimistic, or longer timeframe, models of extinction
	probabilities, is due to the higher probabilities of extinction for non-threatened
	species in these models. This reduces the likelihood of a potentially
	compatible relative surviving.
	An example is if some Critically Endangered species had 2 near relatives,
	both Vulnerable, the Priority Scores
	under each model would be 0.97, 0.85 and 0.36 for the 50 year, 500 year and
	pessimistic models respectively (see Equation \ref{equation}). These results
	show that expanding
	the capability of cross-species cloning approaches to greater phylogenetic
	distances would vastly increase the number of species for which this would be
	a viable conservation approach, particularly in scenarios of continually
	accelerating rates of extinction, or over particularly long time frames. 
	
	Will discuss the restoration simulations later after new results
	
	Optimising conservation spending for a metric of choice
	(in this case simply number of extinctions)
	has a large effect on the cost-effectiveness. The shape
	of the relationship between spending and number of extinctions for the optimised
	approach is curved, because it begins with the species of greatest
	cost effectiveness, making the slope steep, and as it begins to spend on
	less cost-effective species, the slope of the curve lessens. The random approach
	has a linear relationship, as species are sampled equally
	from across the cost effectiveness distribution for any given level of spending.
	Even the optimised approach requires a high level of spending,
	\$900 million over 10 years, to have a noticeable
	effect on the number of extinctions compared to the baseline stochastic variation.
	This level of spending becomes even more significant when considered that
	this data is for only one class, birds, and other classes would likely require
	a similar outlay. There are possible metrics by which this
	could be an overestimate of the actual spending required, as some
	items such as habitat restoration could affect the prospects
	of more than one species
	\citep{mccarthyFinancialCostsMeeting2012}. However, it is very unlikely that
	conservation spending would be optimised in this way in the real world, as
	conservation spending is rarely allocated based solely on a cost-for-value basis
	based solely on extinction risk. Thus, the \$900 million estimate
	seems like a good lower bound for the real-world spending required
	to noticeably affect extinctions.
	The amount of spending required for biobanking compares
	favourably with these estimates. Based on informal conversations with staff at the
	Molecular Collections Facility at the Natural History Museum, material costs
	of biobanking are low, such that even storing a sufficient sample of all bird
	species (usually considered to be 10 samples each for 50 individuals
	\citep{harwoodDevelopingImplementingPrioritisation2021}) would be
	a fraction of the cost of the required conservation spending to reduce
	extinctions. The more significant portion of the costs of cloning-based
	restoration approaches would lie in sample collection and costs at the
	stage of cloning. Since many studies are already taking samples of our
	biodiversity for their own purposes, this presents an opportunity to ensure
	that the unused samples are stored with an eye to long-term
	preservation.
	
	Attempts to draw generalisable conclusions from this study are limited by a
	number of factors. One of these is that, as in the case of the black-footed
	ferret, cross-species restoration is likely to be most practical and
	effective when used with populations that are close to, but have not yet
	undergone extinction. Cross-species cloning and other methods can then be
	used to reintroduce genetic diversity to a population which has entered into
	the extinction vortex, where inbreeding causes rapid loss of diversity and
	fixation of deleterious traits. It would have been difficult to create a
	Priority Score for species based on their entry into an extinction vortex.
	It is currently difficult to predict when species
	will enter an extinction vortex. It has only having been performed in a few cases
	with extant populations \citep{williamsextinctionvortex2020}, and quantitative
	models for prediction have only recently been developed
	\citep{nabutanyiModelsEcoEvolutionaryExtinction2021}. Therefore, I think that
	basing the analysis on performing restoration methods after extinction
	is well-supported, as
	this can in any case serve as a proxy for a species entering the extinction
	vortex.
	
	Another potential source of bias in our estimates lies within species
	classification in the phylogenetic trees. There will always be debate whether
	to classify closely related taxa as separate species or only subspecies. This
	can create a source of bias, particularly if there is a tendency within certain
	genera or orders to classify as species vs. subspecies and vice-versa. Since it
	is necessary for closely related species to survive, artificially inflating the
	number of closely related taxa will bias the priority scores. However, this
	effect should be evened somewhat by the fact that each of these species will 
	have their own extinction risk, which will be incorporated into the
	calculations.
	
	Furthermore, this analysis assumes that the potential use of samples in species
	restoration is the primary aim of any biobanked material, unlike other
	methods for prioritisation such as
	\citep{harwoodDevelopingImplementingPrioritisation2021,mooneyValueExSitu2021}.
	This is a limited perspective on the use of biobanked samples, which may be
	useful for basic scientific research, for other conservation approaches, or
	for scientific applications which are invented in the future. This
	analysis also fails to take into account aspects such as the current number
	of stored samples for a given species, which would be important for
	prioritisation as we can then focus our efforts on under-represented species.
	The insights gained from this more focused analysis can however be used
	to inform prioritisation efforts when considered alongside other important
	aspects such as those discussed above.
	
	
	
	There are also ethical considerations regarding prioritisation and inter-species
	restoration methods.
	Any attempt to create metrics to prioritise choice of species, even in areas
	where this is well-established such as conservation prioritisation via methods
	like EDGE scores \citep{isaacMammalsEDGEConservation2007}, can be controversial,
	partly because it seems distasteful to have to choose between saving different
	species \citep{bottrillConservationTriageJust2008},
	and also because there are so many potential measures of species worth,
	such as cultural worth, ecosystem contribution or evolutionary diversity.
	Cross-species restoration methods are also controversial in and of themselves,
	as it can be argued that you are not truly recreating the target species since
	another species mtDNA is used. This raises questions about whether we should
	be using our resources to create such genetic hybrids, as opposed to spending
	these resources on more standard conservation methods.
	
	Acknowledging the difficulty of resolving such ethical and moral considerations,
	and of the scientific difficulty of expanding inter-species cloning approaches
	beyond very close relatives, we are presented with a rapidly
	closing window to collect samples of many of these endangered species before
	they unfortunately become extinct. This seems like a low-risk venture, as the
	material cost of sample storage is relatively inexpensive. These samples would
	also be useful for many other applications, even if inter-species restoration
	proves unviable or is ruled out for ethical reasons. My results
	in the conservation spending and restoration spending simulations show that
	large increases in the efficacy of conservation efforts can be made, if attempts
	are made at optimisation and criteria for doing so can be
	agreed. Spending on biobanking and cloning-based restoration methods can
	should be thought of as a complement to, rather than a replacement of,
	traditional conservation spending. It may even allow current conservation
	spending to have greater leverage, by improving aspects of population health
	such as genetic diversity.
	Finally, greater coordination among scientists and existing biobanks
	can improve their contribution, as we can ensure that the samples existing in
	biobanks reflect the diversity of life, and that samples taken are stored with
	a view to the long-term, to avoid losing these precious resources.
	
	\section{Acknowledgments}
	First and foremost, I would like to thank my primary supervisor, Dr. James
	Rosindell, for all of his help and support. I would also like to thank
	Prof. Francisco Pelegri at the University of Wisconsin Madison for very
	interesting discussions and for generating the initial idea for this
	project with James. Thanks to everyone at the Molecular Collections
	Facility at the Natural History Museum, and from the Frozen Arks and
	Cryo Arks teams, for their help and discussions. Particularly, thanks to
	Jacqueline Mackenzie-Dobbs for giving us tour of the MCF and providing
	insights into the day-to-day running of a biobank facility.
	
	\section{Appendix}
	\subsection{Bird Downlisting Conservation Expenditure Estimates}\label{expen}
	\csvreader[tabular=|c|c|c|,
		table head=\hline Species & Status & Required Expenditure\\\hline,
		late after line=\\\hline]%
	{cleanedexpen.csv}{"Species"=\species,"Status"=\status,"Required_expen"=\required}%
	{\thecsvrow & \status & \required}%
	
	\bibliographystyle{apalike}
	\bibliography{bibtex,manual}
\end{document}